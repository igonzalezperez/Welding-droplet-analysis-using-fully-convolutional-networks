\begin{abstract}
{Recientemente, modelos de aprendizaje profundo han obtenido excelentes resultados en tareas como clasificación de imágenes, detección de anomalías y procesamiento de lenguaje. Estos modelos mejoran cuando hay una gran cantidad de datos, lo que es común hoy en día. Sumado al aumento en poder de cómputo de las GPU, es posible resolver problemas que de otro modo serían intratables. No obstante, el enfoque de aprendizaje profundo se ha aplicado principalmente en visión computacional, mientras que en otros campos de la ciencia no se ha aplicado al mismo nivel. De este modo, existe una oportunidad de usar estos modelos y mejorar resultados en el campo de la soldadura.

En el siguiente trabajo el propósito es utilizar modelos de aprendizaje profundo para obtener características relevantes de un proceso de soldadura a gas y arco metálico (GMAW). Lo anterior consiste en segmentar imágenes para separar a las gotas del fondo y luego calcular parámetros. Este problema ha sido abordado con el uso de técnicas de visión computacional, sin embargo las técnicas usadas carecen de automatización, por lo que procesar una gran cantidad de datos se hace inviable. El enfoque de esta tesis permite resolver el problema de segmentación en sí, y el problema de automatización, pues los resultados se pueden obtener rápidamente.

El modelo propuesto se basa en Fully Convolutional Networks, las cuales mediante entrenamiento supervisado reciben imágenes y entregan una máscara de segmentación. Luego, las máscaras se usan para calcular propiedades geométricas y físicas, lo cual es de ayuda para comprender y diseñar el proceso de soldadura, ya que permite establecer una relación entre input como voltaje, corriente o gas protector y output como posición, área, frecuencia o velocidad de las gotas.

Se realiza una revisión de literatura para comprender el problema y cómo se ha abordado hasta ahora, y para estudiar propuestas de segmentación con aprendizaje profundo. Luego, se recopilan datos para un caso de estudio, los datos consisten en videos de modos de transferencia globular y spray que luego son manualmente etiquetados. Posteriormente se entrenan modelos Fully Convolutional con los datos etiquetados, en particular se prueban las arquitecturas U-Net, DeconvNet y MultiResUnet. Finalmente, las máscaras de segmentación se utilizan para calcular características importantes.

La conclusión de este trabajo es que la propuesta de un modelo tipo U-Net es capaz de segmentar correctamente los videos, consiguiendo máscaras similares a propuestas anteriores, pero con la ventaja de poder procesar miles de imágenes en segundos o minutos. Además, se obtiene la posición, velocidad, aceleración, perímetro área, volumen, tensión superficial y tasa de desprendimiento mayormente en concordancia con la literatura. Por lor tanto, este trabajo supone un exitoso paso hacia la automatización y la mejor comprensión de GMAW.}
\end{abstract}
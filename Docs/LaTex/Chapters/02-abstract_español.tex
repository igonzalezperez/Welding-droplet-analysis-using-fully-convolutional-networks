\begin{abstract}
{Recientemente, modelos de aprendizaje profundo han obtenido buenos resultados en tareas como clasificación de imágenes, detección de anomalías y traducción de idiomas. Estos modelos mejoran cuando hay una alta cantidad de datos, lo que es común hoy en día. Sumado al constante aumento en poder de cómputo de las GPU, es posible resolver problemas que de otro modo serían difíciles o intratables. Sin embargo, el enfoque de aprendizaje profundo se ha aplicado principalmente en visión computacional, mientras que en otros campos de la ciencia no se ha aplicado al nivel que los avances permitirían. De este modo, existe una oportunidad de usar estos modelos y mejorar resultados en nuevos campos del conocimiento. Lo anterior ha ocurrido en algunas disciplinas como lo son la biomedicina, gestión de activos físicos o la química, con excelentes resultados.

En el siguiente trabajo el propósito es utilizar modelos de aprendizaje profundo de segmentación para obtener características relevantes de un proceso de soldadura a gas y arco metálico (GMAW), esto se hace mediante la segmentación de un video para separar a las gotas del fondo y luego calcular características. El problema ha sido abordado con el uso de técnicas de visión computacional, pero las técnicas usadas carecen de automatización, por lo que procesar una gran cantidad de datos se hace inviable. Por lo tanto, el enfoque de esta tesis permite resolver el problema de segmentación en sí, y el problema de automatización, pues los resultados se pueden obtener rápidamente.

El modelo propuesto es una red neuronal tipo U-Net, la cual mediante entrenamiento supervisado recibe imágenes y entrega una máscara de segmentación. Luego, las máscaras se usan para calcular propiedades geométricas y cinemáticas, lo cual es de ayuda para comprender y diseñar el proceso de soldadura, ya que permite establecer una relación entre input como voltaje, corriente o gas protector y output como posición, área, frecuencia y velocidad de las gotas.

La metodología es la siguiente: Primero, se realiza una revisión de literatura para comprender el problema y cómo se ha abordado hasta ahora, y para estudiar propuestas de segmentación de aprendizaje profundo. Segundo, se adquieren datos para un caso de estudio, los datos consisten en videos de procesos de GMAW con modos de transferencia globular y spray. Los videos se separan en cuadros y una porción de ellos es etiquetada. Posteriormente se entrena un modelo tipo U-Net con los datos etiquetados. Tercero, las máscaras de segmentación se utilizan para calcular características como posición, área, velocidad, frecuencia de deposición y frecuencia de oscilación. Finalmente, los resultados se discuten en términos de su confiabilidad y aplicablidad en soldadura.

La conclusión principal de este trabajo es que la propuesta de un modelo tipo U-Net es capaz de segmentar correctamente las gotas dentro de una imagen, consiguiendo máscaras similares a propuestas anteriores, pero con el beneficio de poder procesar una miles de imágenes en minutos. Además, el post-procesamiento es capaz de obtener propiedades físicas del proceso que permiten entenderlo y diseñarlo mejor.}
\end{abstract}
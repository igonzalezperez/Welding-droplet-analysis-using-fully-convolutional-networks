\begin{abstract}
{Recently, deep learning models have performed well in tasks such as image classification, anomaly detection and language translation. These models thrive when the amount of data is high, which is common nowadays. Coupled with the increasing computing power of GPU it is possible to solve many of otherwise difficult or intractable problems. Nevertheless, the deep learning approach has been mainly used in computer vision while other scientific fields have not applied it as much as the advances would allow. Hence, there is an opportunity to use these models to outperform previous approaches in novel fields. This has happened in some fields such as bio-medicine, prognostics and health management and chemistry with great success.

In the following work the aim is to use deep learning segmentation models to solve the problem of obtaining relevant features of a Gas Metal Arc Welding (GMAW) process, this is done by segmenting video footage to isolate droplets and then calculating features that are relevant to characterize the process. This problem has been addressed before with computer vision techniques, but the methods used lack automation, and processing large volumes of data becomes unfeasible. Therefore, this thesis' approach allows to achieve results in the segmentation problem itself and the automation process, since results can be obtained faster.

The proposed model is a fully convolutional network approach. Several architectures are considered and compared to then use the best one for further calculations, which by means of supervised training can take a large amount of images and return segmentation masks for each one, isolating the droplets from the background. Furthermore, segmentation masks are used to compute geometric and kinematic properties. This can be helpful to understand and design the welding process, since it would be possible to map the inputs like current, voltage and shielding gas to a measurable output such as position, area, frequency and velocity of droplets.

The methodology is as follows: First, a literature review is done to understand the problem and how it has been addressed so far and to study the approaches for segmentation in deep learning. Second, data is acquired for a study case, the data consists of two videos of GMAW processes with depicting globular and spray transfer. The videos are split into frames and then a small sample is manually labeled. Furthermore, fully convolutional network based model are trained with the labeled data, namely U-Net, DeconvUnet and MultiResUnet. Third, the resulting segmentation masks are processed to compute features such as position, area, velocity,  deposition frequency and oscillation frequency. Finally, the results are discussed in terms of the reliability of the results and the applicability of a deep learning framework in the welding droplet segmentation task.

The main conclusion of this work is that the U-Net based approach can reliably segment droplets within a frame, achieving similar results to previous attempts, but with the benefit of being able to process thousands of images in minutes. Furthermore, the features measured in the post-processing can be used to further understand the process.}
\end{abstract}